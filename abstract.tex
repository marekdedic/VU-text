\begin{czech}
	\begin{onehalfspace}
		\noindent \textit{Název práce:}

		\noindent \textbf{Hluboké neuronové sítě pro problémy s hierarchií}
	\end{onehalfspace}

	\bigskip

	\noindent \textit{Autor:} Marek Dědič

	\bigskip

	\noindent \textit{Obor:} Matematická informatika

	\bigskip

	\noindent \textit{Druh práce:} Výzkumný úkol

	\bigskip

	\noindent \textit{Vedoucí práce:} Ing. Tomáš Pevný, Ph.D., Cisco systems, Inc.

	\bigskip

	\noindent \textit{Abstrakt:}
	Algoritmus zpětné propagace je v praxi standardním přístupem pro výpočet gradientů pro vrstevnaté neuronové sítě. Je však nepravděpodobné, že by se tento algoritmus mohl vyskytovat v biologických neuronech. Tato práce vychází z nedávných pokroků ve výpočetní neurovědě a zkoumá algoritmus target propagation jako biologicky inspirovanou alternativu ke zpětné propagaci. Cílem této práce bylo posoudit použitelnost tohoto algoritmu, zvláště pak v kontextu multi-instančního učení. Oba již zmiňované algoritmy byly popsány a použity na jednoduchý problém. Výsledky práce jsou převážně negativní. Target propagation, tak jak byla implementována, není použitelnou alternativou ke zpětné propagaci. Tento problém byl hlouběji zkoumán a byly analyzovány jeho možné příčiny. S ohledem na špatný výkon target propagation nebyla dále zkoumána její aplikovatelnost na multi-instanční problémy.

	\bigskip

	\noindent \textit{Klíčová slova:}
	strojové učení, multi-instanční učení, zpětná propagace, target propagation, difference target propagation, credit assignement problem, dataset dvou půlměsíců, neuronové sítě, vícevrstvý perceptron
\end{czech}

\vfill

\begin{onehalfspace}
	\noindent \textit{Title:}

	\noindent \textbf{Deep neural networks for problems with hierarchy}
\end{onehalfspace}

\bigskip

\noindent \textit{Author:} Marek Dědič

\bigskip

\noindent \textit{Abstract:}
The back-propagation algorithm is the \textit{de facto} standard approach to computing gradients for feedforward neural networks. It is, however, implausible that this algorithm could be used in biological neurons. This work builds on recent advances in computational neuroscience and explores the target propagation algorithm as a biologically inspired alternative. The aim of this work was to explore the viability of this algorithm, particularly for multi-instance learning. Both the aforementioned algorithms are described and then evaluated on a simple problem. The results are largely negative. Target propagation as it was implemented is not a viable alternative to back-propagation. A deeper exploration of the issue is included, followed by a discussion of the possible causes of such issues. Due to the poor performance of target propagation, its applicability to multi-instance problems wasn't further explored.

\bigskip

\noindent \textit{Keywords:}
machine learning, multi-instance learning, back-propagation, target propagation, difference target propagation, credit assignment problem, two-moon dataset, neural networks, multi-layer perceptron
